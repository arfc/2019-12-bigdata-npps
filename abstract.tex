\documentclass[10pt,a4paper]{article}

\usepackage{graphicx}
\usepackage[left=2.5cm,right=2.5cm,top=2.5cm,bottom=2.5cm]{geometry}
\usepackage{geometry}
\usepackage{pdflscape}
\usepackage{amsmath}
\usepackage{amsfonts}
\usepackage{amssymb}
\usepackage{microtype}

\usepackage[hidelinks]{hyperref}
\usepackage{cleveref}

\bibliographystyle{elsarticle-num}

\begin{document}
\title{Machine-Learning-Enabled Design for Molten Salt Reactors}
\author{Mehmet Turkmen$^1$, Kathryn D. Huff$^1$\\
$^1$Dept. of Nuclear, Plasma, and Radiological Engineering, \\University of Illinois at Urbana-Champaign, Urbana, IL 61801, United States}

\maketitle


\section*{Abstract}
How to determine the design parameters of a nuclear reactor that would yield 
more economical and safer design, and has higher breeder capability and greater 
operational flexibility is an important challenge in the field of nuclear 
engineering. In order to cope with this issue, an efficient, robust and 
reliable method which is based on the use of the machine learning tools of 
artificial intelligence technique is suggested in this study. The suggested 
method is applied to the single fuel channel of Molten Salt Fast Reactor for 
the selected seven independent input parameters: fuel (U, U/Pu, U/Th) and salt 
(Na, Cl, F) types, enrichment, pitch and length of the channel, 
moderator-to-salt ratio and channel power (or channel outlet temperature). The 
outputs to be optimized are the infinite multiplication factor, conversion 
ratio, fast flux on graphite material and feedback coefficients. The tools 
(optimization, data mining and sampling etc.) of RAVEN framework, developed by 
the Idaho National Laboratory, is used to produce the datasets, to explore the 
search space and to train a model for the prediction of anticipated datasets. 
The framework is coupled with SERPENT code as an external Monte Carlo particle 
transport code to compute the defined outputs. This research would make a 
significant contribution to the efforts on the improvement of the core design 
by exposing all possible scenarios, facilitating the decision making, 
accelerating the designing process and reducing the man-hours consumed in 
modeling.


\end{document}
