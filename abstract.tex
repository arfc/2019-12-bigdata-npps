\documentclass[10pt,a4paper]{article}

\usepackage{graphicx}
\usepackage[left=2.5cm,right=2.5cm,top=2.5cm,bottom=2.5cm]{geometry}
\usepackage{geometry}
\usepackage{pdflscape}
\usepackage{amsmath}
\usepackage{amsfonts}
\usepackage{amssymb}
\usepackage{microtype}

\usepackage[hidelinks]{hyperref}
\usepackage{cleveref}

\bibliographystyle{elsarticle-num}

\begin{document}
\title{Machine-Learning-Enabled Design for Molten Salt Reactors}
\author{Mehmet Turkmen$^1$, Kathryn D. Huff$^1$\\
$^1$Dept. of Nuclear, Plasma, and Radiological Engineering, \\University of Illinois at Urbana-Champaign, Urbana, IL 61801, United States}

\maketitle


\section*{Abstract}
Nuclear reactor design can improve economics, safety, breeding capability, and 
operational flexibility. 
Accordingly, we suggest an efficient, robust, and reliable design recommendation 
approach based on machine learning methods.
This approach explores seven independent input parameters
in the context of a  single fuel channel of Molten Salt Fast Reactor, including 
fissile material (U, U/Pu, U/Th), salt type (Na, Cl, F), enrichment, channel 
pitch, channel length,
moderator-to-salt ratio, and channel power (or channel outlet temperature). 
Performance metric for optimization include 
$k_{inf}$, the conversion ratio ($\gamma$), feedback coefficients ($\alpha_D$, 
$\alpha_{V}$, $\alpha_{M}$), and the fast flux incident on graphite material.
Optimization, data mining and sampling in this work will rely on the RAVEN 
framework, developed by the Idaho National Laboratory, to produce the datasets, 
to explore the parameter space, and to train a model for recommending design 
optimization. The tool created in this work couples to the SERPENT neutron transport software.  
to compute these design performance metrics. We expect this tool will make
significant contributions to core design optimization by exposing multitudinous 
possible scenarios, facilitating decision making, accelerating the design 
process and reducing the man-hours consumed in modeling.


\end{document}
